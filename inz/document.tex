\documentclass[eng,printmode,oneside]{mgr}
%opcje klasy dokumentu mgr.cls zostaly opisane w dolaczonej instrukcji

%ponizej deklaracje uzycia pakietow, usunac to co jest niepotrzebne
\usepackage{polski} %przydatne podczas skladania dokumentow w j. polskim
%\usepackage[polish]{babel}%alternatywnie do pakietu polski, wybrac jeden z nich
\usepackage[utf8]{inputenc} %kodowanie znakow, zalezne od systemu
\usepackage[T1]{fontenc} %poprawne skladanie polskich czcionek

%listingi
\usepackage{listings}
\usepackage{color}
\definecolor{zielony}{rgb}{0,0.6,0}
\definecolor{innyZielony}{rgb}{0.5,0.7,0}
\renewcommand{\lstlistingname}{Listing}% Listing -> Algorithm
\renewcommand{\lstlistlistingname}{List of \lstlistingname s}% 

%kolor do komentarzy, uwag, przypomnienie
\definecolor{komentarz}{rgb}{1,0,0}

%pakiety do grafiki
\usepackage{graphicx}
\usepackage{subfigure}
\usepackage{psfrag}

%pakiety dodajace duzo dodatkowych polecen matematycznych
\usepackage{amsmath}
\usepackage{amsfonts}

%pakiety wspomagajace i poprawiajace skladanie tabel
\usepackage{supertabular}
\usepackage{array}
\usepackage{tabularx}
\usepackage{hhline}

%pakiet wypisujacy na marginesie etykiety rownan i rysunkow zdefiniowanych przez \label{}, chcac wygenerowac finalna wersje dokumentu wystarczy usunac 
%ponizsza linie
\usepackage{showlabels}

%definicje wlasnych polecen
\newcommand{\R}{I\!\!R} %symbol liczb rzeczywistych, dziala tylko w trybie matematycznym
\newtheorem{theorem}{Twierdzenie}[section] %nowe otoczenie do skladania twierdzen

%dane do zlozenia strony tytulowej
\title{System monitoringu lokalizacji przesyłek kurierskich}
\engtitle{English title ŁŁĄŚŹĆŻÓ łąśćźżóę Ę}
\author{Monika Strachowska}
\supervisor{dr hab. inz. Imie Nazwisko Prof. PWr, I-6}
%\guardian{dr hab. inz. Imie Nazwisko Prof. PWr, I-6} %nie uzywac jesli opiekun jest ta sama osoba co prowadzacy prace

%\date{2008} %standardowo u dolu strony tytulowej umieszczany jest biezacy rok, to polecenie pozwala wstawic dowolny rok

%ponizej jest lista kierunkow i specjalnosci na wydziale elektroniki, nalezy wybrac wlasciwe lub dopisac jesli nie ma odpowiednich
\field{Automatyka i Robotyka (AIR)}
\specialisation{Systemy informatyczne w automatyce (ASI)}

%tutaj zaczyna sie wlasciwa tresc dokumentu
\begin{document}

\lstdefinestyle{listJava}{language=Java, breaklines=true,
commentstyle=\color{zielony},frame=single, numbers=left, stepnumber=1,
numbersep=5pt}
\lstdefinestyle{listPython}{language=Python, breaklines=true,
commentstyle=\color{innyZielony},frame=single, numbers=left, stepnumber=1,
numbersep=5pt}

\bibliographystyle{plabbrv} %tylko gdy uzywamy BibTeXa, ustawia polski styl bibliografii

\maketitle %polecenie generujace strone tytulowa


\tableofcontents %spis tresci

%ponizej znajduje sie przykladowa tresc dalszej czesci dokumentu, zainteresowanych zachecam do rozszyfrowania frazy "Lorem ipsum" :)
\chapter{Wstęp i cel pracy}

Współcześnie coraz więcej osób korzysta z możliwości zakupów przez internet,
niesie to za sobą wiele korzyści. Często zakupiony towar jest tańszy, unikalny,
bądz niedostępny w stacjonarnym sklepie czy poprostu jest to wygodniejsza forma
zakupów. Oprócz wyżej wymienionych zakupów itotnym twoarem przewożonym
są dokumenty, często szybko potrzebne. Z tych powodów ludzie zamawijący usługi 
kurierskie chcieliby dostać jak najwyższą jakość. Zwiększenie jakości tej
usługi może nastąpić poprzez skrócenie czasu dostarczenia przesyłki(co
fizycznie już jest nie osiągalne), tańszy jej koszt, czy na przykład możliwość
sprawdzenia, w jakim dokładnie miejscu ona się znajduje. Taka funkcjonalność 
usługi kurierskiej nie należy do jakości wymaganej i koniecznej, ale znacznie
poniesie popyt i prestiż firmy kurierskiej, która zdecyduje się na taką
dodatkowa funkcjonalność. Z punktu widzenia klienta odczuwany jest komfort
informacji, gdzie jest przesyłka, dzięki temu klient może zaplnaować sobie
dzień, w którym nastąpi dostarczenie. 

Przedstawiany tu projekt rozwija aktualną funkcjonalność firm kurierskich o
graficzne przedstawienie, w formie mapy, aktualnej lokalizacji przesyłki. Taka
forma prezentacji jest prosta w odbiorze i dużo bardziej czytelna niż wyniki
jakie prezentowane są aktualnie w formie tabel, w których zawarte są miejsca
odbicia przesyłki. Ponadto praca próbuje rozwiązać problem jaki istnieje w
estymacji czasu dostraczaniu przesyłki do adresata, esytmacja czasu
dostarczenia jest bardzo niedokładna(ogólna) lub jej nie ma.

\emph{\color{komentarz}
Projekt został zrealizowany z wykorzystaniem takich
technologi jak:
język programowania Java, system operacyjny Android, baza danych MySQL, Google Apps.
http://www.lokalizacja.info/pl/testy/monitoring/gdzie-jest-moja-paczka-test-firm-kurierskich.html\#.VEzCPVS988o
Firmy kurierskie maja najprawdopodobniej system „windows ce/mobile” na swoich urządzenia. Ja ze względu na brak takiego urządzenia (mobilnego z windowsem) 
zrealizuje zadanie na androidzie.
}

\begin{figure}[ht!]
\centering
\includegraphics[width=90mm]{obr.jpg}
\caption{podpisisi}
\end{figure}
\chapter{Rozwiązanie - prezentacja wyników}


Nie tylko gps do lokalizacji, bo także odbicia na czytnikach u kurierów 
Projekt opiera się 

schemat - wejscie (adnroid) - środek system - wyjscie www z mapka

Tu udaje kod
\lstset{style=listJava}
\begin{lstlisting}[caption={to jest podpis}]
class Srass {
	public kupaGowna() {
	} //takakaka
}
\end{lstlisting}
\lstset{style=listPython}
\begin{lstlisting}[caption={to jest podpis drugiego}]
class Srass {
	public kupaGowna() {
	} //takakaka
}
\end{lstlisting}

Udałam kod
\chapter{Wykorzystane technologie}

Zrealizowany tu projekt bazuje na nowoczesnych technologiach.
Skorzystano z mobilnego urządzenia - telefonu komórkowego z systemem
Android, bazy danych do przechowywania inforamcji, a także serweru, który to
łączy wszystkie elementy w jedną spójną całość. Głównym językiem
programowania wykorzystanym w projekcie jest język Java, dzięki któremu
zrelizowano aplikację mobilną, obsługę servletu, bazy danych, odpytywania i
parsowania odpowiedzi serwera Google o widok mapy i odległośći pomiędzy dwoma
punktami, a także obsługa witryny http. Całą oplikację stworzono za pomocą IDE
Eclipse z odpowiednimi dodatkami.
\emph{\color{komentarz}logo Javy}

\section{Java}

Java jest obiektowym językiem programowania ogólnego przeznaczenia.
Charakteryzuje się silnym ukierunkowaniem na obiektowość oraz niezależnością i
przenoszalnośćią kodu od architekty.Oprócz wyżej wymieniowych założeniami języka
Java jest prostota, sieciowość, niezawodność, bezpieczność, interpretowalność, wysokowydajny,
wielowątkowy, dynamiczny oraz niezależny od architektury. 
\begin{itemize}
  \item Prosty - założeniami autorów języka Java było aby programista bez
  specjalnych szkoleń mógł od razu zacząć pisać w języku Java. Skłądnia została
  oczyszczona (w stostunku do C++) o arytmetykę wskaźnikową, struktury, unie,
  przeciążanie operatorów itd.
  \item Zorientowany obiektowo
  \item Sieciowy - Java poasiada bbiliotekę, która w przystępny sposób umożlwia
  pracę z protokałami http, TCP/IP, ftp
  \item Niezawodny - szczególnie skupiono się na wykrywaniu ewentualnych
  problemów, zapobieganiu sytuacjom, w których może błąd nastąpić oraz
  sprawdzaniu błedów podczas działania programu
  \item Bezpieczny - Java może służyć do zastosowań sieciowych, z tego powodu
  zadbano o możwlie najlepsze zabezpieczenie przed wirusami i ingerexją osób
  trzecich.
  \item Niezależny od architektury - Java kompilowana jest do kodu pośredniego
  (bajtoweg), który następnie jest interpretowany na maszynie wirutalnej Javy,
  która jest dostosowana do odpowiedniego systemu. Maszyna wirtualna Javy(JVM)
  jest zdolna wykonywać program z kodu pośredniego. Z tego powodu jeżyk Java
  stosowany jest na wielu urządzeniach oraz różnych systemach operacyjnych.
  Niestyty konsekwencją przenoszalności kodu jest jego wolniejsze wykonanie.
  \item Przenośny - Java posiada ściśle określone rozmiary typów danych i nie ma
  możliwości zmiany rozmiaru przez proramiste przez co nie następuje np. zmiana
  kolejności bajtów
  \item Interpretowany - \ldots
  \item Wysokowydajny - istnieje możliwość tłumaczenia kodu bajtowego w locie,
  co zwiększa szybkość ładowania się programu
  \item Wielowątkowy - pozwala na interaktywność między procesami, a także pracę
  w czasie rzeczywistym
  \item Dynamiczny - obiekty w Javie można zmieniać w zależności od
  zmieniającego się środowiska oraz możliwy jest wgląd we wszystkie obiekty, a
  nawet dodawać nowe metody
\end{itemize}

Język Java wywodzi się z języków C++ i C, wykorzystuje wiele potrzebnych i
użytecznych funkcjonalności tych języków, z nieużytecznych, trudnych lub
pwoowdujących często błędy zrezygnowano. Język Java umożliwia dziedziczenie, a ponadto wszystkie obiekty Javy
są pochodną obiektu bazwego. Jednakże Java nie umożliwa dziedziczenia wielobazowego, dlatego do Javy wprowadzono interfejsy - abstrakcyjny typ, który posiada jedynie opracje, ale
nie posiada danych, z tego powodu można tylko implementować interfejs i nie
można utworzyć obiektów tego typu. Język Java umożliwia pisanie aplikacji
stacjonarnych, webowych czy mobilnych. Język Java ma rozbudowaną obsługę
wyjątków. Posiada dobrze rozbudowanego GarbageCollector (odśmieciacza)
\cite{java.doc}.
\subsection{podpodrozdział}

\section{Servlet}

Serwlety są to aplikacje działające na serwerze WWW korzystające z języka Java.
Sewrwlety mają zapewniać budowanie aplikacji internetowych niezależnych od
platformy. Serwlet umożliwia korzystanie z baz danych i http. Z tego powodu
wykrozystywane są do budowania interaktywnych aplikacji internetowych. 

W projekcie skorzystano z serwletu Tomcat Apache, który jest open source'owy i
korzysta z licencji Apache. Serwer Apache obsługuje www za pomocą protokołu
http, jest otwarty, zapewnia wielowątkowość, skalowalność, bezpieczeństwo,
kontrolę dostępu \cite{apache.wiki}. \emph{\color{komentarz} tu obrazek apache i
tomcat \cite{apache.tomcat}}

\section{Technologie internetowe}

W przedstawionym w tej pracy projekcie korzystano z technologi internetowych,
które obsługiwały interakcję z użytkownikiem oraz `strony www`. Skorzystano
z takich technologii jak:
\begin{itemize}
  \item JavaScript - jest to skryptowy jezyk programowania stosowany do
  tworzenia stron internetowych, zapewnia interakcję z użytkownikiem\cite{javascript.wiki}
  \item XML - jesto to język znaczników przeznaczony do reprezenowania danych w
  strukturyzowany sposób \cite{xml.wiki}
  \item Protokół http - 
  \item \ldots
\end{itemize}
\section{db - MySQL}

W projakcie do przechowywania danych skorzystano z baz danych. Baza danych
pozwala w ustrukturyzowany sposób kolekcjonować dane niezbędnę do działania
programów. Przechowywane dane mogą być o dwolnonym formacie i strukturze.
Systemem, który zarządzał bazą danych w projekcie był MySQL.

\section{Android}

Android jest systemem wykorzystywanym na platformach mobilnych. Android
jest systemem operacyjnym z rodziny Linux, oparty na jądrze Linux. Android
umożliwia tworzenie aplikacji na wiele urządzeń, optymalizacji podlega plik xml,
gdzie można dostosować aplikację do konkretnych urządzeń. Najnowższą wersją
systemu jest Android Lollipop 5.0. 

Rozpoczęcie pracy z Androidem zaczyna się od instalacji środowiska, może to być
Eclipse z dodatkiem SDK Android lub Android Studio. A samo tworzenie aplikacji
od projektu interfejsu użytkownika, następnie dopiero oprogramowuje się obsługę
oraz logikę aplikacji, ostatnim etapem jest testowanie aplikacji.
\cite{developer.android}
\emph{\color{komentarz}napisac o tym, ze android oddelegowuje zadania, taki
obrazek ze strony}

\section{Google apps - mapy}

Firma Google udostępnia korzystanie deweloperom ze swoich produktów
\cite{developer.google}. W swoim projekcie korzystałam z Google Maps Api. 

\chapter{Możliwość rozwinięcia w przyszłości}
Aplikację można „podpiąć” pod prawdziwe urządzenia jakie posiadają kurierzy – te na których się człowiek podpisuje – ale konieczne będzie 
zrefakturyzowanie(?)/zmienie kodu pod system, który mają tam zainstalowany.
	Fajnie by było to wrzycić na prawdziwe tablety, można by sprzedawać/zarobić. Ogólnie koszt takiego urządzenia to byłoby tablet/telefon 
	+ wycena za program.
	Normalnie kurierzy urzywają kolektorów danych.

\chapter{Wnioski i podsumowanie}
\addcontentsline{toc}{chapter}{Bibliografia} %utworzenie w spisie tresci pozycji Bibliografia
\bibliography{bibliografia} % wstawia bibliografie korzystajac z
% pliku bibliografia.bib - dotyczy BibTeXa, jezeli nie korzystamy z BibTeXa nalezy uzyc otoczenia thebibliography

%opcjonalnie moze sie tu pojawic spis rysunkow i tabel
\listoffigures
\listoftables
\lstlistoflistings
\end{document}
