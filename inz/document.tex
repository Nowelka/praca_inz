\documentclass[eng,printmode,oneside]{mgr}
%opcje klasy dokumentu mgr.cls zostaly opisane w dolaczonej instrukcji

%ponizej deklaracje uzycia pakietow, usunac to co jest niepotrzebne
\usepackage{polski} %przydatne podczas skladania dokumentow w j. polskim
%\usepackage[polish]{babel}%alternatywnie do pakietu polski, wybrac jeden z nich
\usepackage[utf8]{inputenc} %kodowanie znakow, zalezne od systemu
\usepackage[T1]{fontenc} %poprawne skladanie polskich czcionek

%pakiety do grafiki
\usepackage{graphicx}
\usepackage{subfigure}
\usepackage{psfrag}

%pakiety dodajace duzo dodatkowych polecen matematycznych
\usepackage{amsmath}
\usepackage{amsfonts}

%pakiety wspomagajace i poprawiajace skladanie tabel
\usepackage{supertabular}
\usepackage{array}
\usepackage{tabularx}
\usepackage{hhline}

%pakiet wypisujacy na marginesie etykiety rownan i rysunkow zdefiniowanych przez \label{}, chcac wygenerowac finalna wersje dokumentu wystarczy usunac 
%ponizsza linie
\usepackage{showlabels}

%definicje wlasnych polecen
\newcommand{\R}{I\!\!R} %symbol liczb rzeczywistych, dziala tylko w trybie matematycznym
\newtheorem{theorem}{Twierdzenie}[section] %nowe otoczenie do skladania twierdzen

%dane do zlozenia strony tytulowej
\title{System monitoringu lokalizacji przesyłek kurierskich}
\engtitle{English title ŁŁĄŚŹĆŻÓ łąśćźżóę Ę}
\author{Monika Strachowska}
\supervisor{dr hab. inz. Imie Nazwisko Prof. PWr, I-6}
%\guardian{dr hab. inz. Imie Nazwisko Prof. PWr, I-6} %nie uzywac jesli opiekun jest ta sama osoba co prowadzacy prace

%\date{2008} %standardowo u dolu strony tytulowej umieszczany jest biezacy rok, to polecenie pozwala wstawic dowolny rok

%ponizej jest lista kierunkow i specjalnosci na wydziale elektroniki, nalezy wybrac wlasciwe lub dopisac jesli nie ma odpowiednich
\field{Automatyka i Robotyka (AIR)}
\specialisation{Systemy informatyczne w automatyce (ASI)}

%tutaj zaczyna sie wlasciwa tresc dokumentu
\begin{document}

\bibliographystyle{plabbrv} %tylko gdy uzywamy BibTeXa, ustawia polski styl bibliografii

\maketitle %polecenie generujace strone tytulowa


\tableofcontents %spis tresci

%ponizej znajduje sie przykladowa tresc dalszej czesci dokumentu, zainteresowanych zachecam do rozszyfrowania frazy "Lorem ipsum" :)
\chapter{Wstęp i cel pracy}
Współcześnie coraz więcej osób korzysta z możliwości zakupów przez internet,
niesie to za sobą wiele korzyści. Często zakupiony towar jest tańszy, unikalny,
bądz niedostępny w stacjonarnym sklepie czy poprostu jest to wygodniejsza forma
zakupów. //Moze cos o tym, ze dokumenty// Z tych powodów ludzie zamawijący
usługi kurierskie chcieliby dostać jak najwyższą jakość. Zwiększenie jakości tej
usługi może nastąpić poprzez szybsze dostarczenie przesyłki, tańszy jej koszt,
czy na przykład możliwość sprawdzenia, w jakim dokładnie miejscu ona się
znajduje. Taka funkcjonalność usługi kurierskiej nie należy do wymaganej



 że wszyscy wysyłają i odbierają i każdy z nich chce jak najszybciej
otrzymać swoją przesyłkę. Niecierpliwią się ludzie a tak widzą gdzie ona jest, jest to dodatkowe rozwinięcie funkcjonalności firmy, 
nie jest konieczne ale może zaplusować. Fajne graficzne, proste w odbiorze przedstawienie danych (bo mapa).
Zainterowanie zakupami przez internet jest duże-ciągle zwiększające się dlatego, że przez internet jest wygotnie i nierzadko taniej. 
Niniejsza praca próbuje rozwiązań problem braku dokładnej lokalizacji przesyłki, a ponadto w znaczny sposób ułatwi estymację czasu otrzymania przesyłki 
przed adresata.  
Największym problemem jest brak lub słaba estymacja czasu kiedy paczka dotrze do adresata.
http://www.lokalizacja.info/pl/testy/monitoring/gdzie-jest-moja-paczka-test-firm-kurierskich.html#.VEzCPVS988o
Firmy kurierskie maja najprawdopodobniej system „windows ce/mobile” na swoich urządzenia. Ja ze względu na brak takiego urządzenia (mobilnego z windowsem) 
zrealizuje zadanie na androidzie.
\chapter{Rozwiązanie - prezentacja wyników}
\chapter{Wykorzystane technologie}
ogólnie: model MVC...
	ogólnie korzystać powinno z bazy danych firmy, to nie zakres tej pracy oraz wyświetlać to w  witrynie, będzie tylko mapka z napisem. Łaczy się z google – 
	pobiera stamtąd mapy/drogi, a z gps pojazdy który ma daną przesyłkę na pokładzie do systemu trafiają jego szerokość i dłg geograficzna i wyznacza się 
	gdzie ona jest.
	Nie tylko gps do lokalizacji, bo także odbicia na czytnikach u kurierów 

\section{Java}
Blala blalla jksajBLABLALBLABLABLALBLA 
dfdsv
\section{JavaScritp}
\section{db}

BLABLALBLA
\section{Android}
BLABLALBLABLABLALBLABLABLALBLABLABLALBLA 

\begin{itemize}
\item Mauris nonummy lorem at orci.
\item Donec accumsan aliquam libero.
\item Donec fringilla ultricies diam.
\item Nulla venenatis est non ligula.
\item Morbi in mi convallis dolor accumsan egestas.
\item Sed euismod nibh in nulla.
\item Sed rhoncus lorem at lectus.
\item Pellentesque fermentum rutrum dui.
\end{itemize}
BLABLALBLA 

Cum sociis natoque penatibus et magnis dis parturient montes, nascetur ridiculus mus. Donec justo diam, auctor vitae, consectetuer sed, aliquet non, lectus. 
Suspendisse id lectus. Integer fermentum metus non felis. Praesent dui augue, auctor non, congue sed, aliquam vel, quam. Fusce et ligula dignissim sapien 
congue scelerisque. Phasellus tincidunt. Aliquam tempus, leo sollicitudin commodo aliquam, neque nisl nonummy ante, ac tristique lacus odio nec libero. Cras 
nec nisl sed erat commodo eleifend. Vivamus tellus. Fusce nec nibh ut neque malesuada feugiat. Nam molestie volutpat lacus. Aliquam sollicitudin nunc at 
turpis. Maecenas lectus quam, aliquam nec, aliquet eget, porta non, massa. Duis viverra nonummy mauris.
\section{Servlet}
pierupierdu
\section{Google apps - mapy}

\chapter{Możliwość rozwinięcia w przyszłości}
Aplikację można „podpiąć” pod prawdziwe urządzenia jakie posiadają kurierzy – te na których się człowiek podpisuje – ale konieczne będzie 
zrefakturyzowanie(?)/zmienie kodu pod system, który mają tam zainstalowany.
	Fajnie by było to wrzycić na prawdziwe tablety, można by sprzedawać/zarobić. Ogólnie koszt takiego urządzenia to byłoby tablet/telefon 
	+ wycena za program.
	Normalnie kurierzy urzywają kolektorów danych.

\chapter{Wnioski i podsumowanie}
\appendix
\chapter{BLABLALBLA}
BLABLALBLABLABLALBLABLABLALBLABLABLALBLABLABLALBLABLABLALBLA
\addcontentsline{toc}{chapter}{Bibliografia} %utworzenie w spisie tresci pozycji Bibliografia
\bibliography{bibliografia} % wstawia bibliografie korzystajac z
% pliku bibliografia.bib - dotyczy BibTeXa, jezeli nie korzystamy z BibTeXa nalezy uzyc otoczenia thebibliography

%opcjonalnie moze sie tu pojawic spis rysunkow i tabel
\listoffigures
\listoftables
\end{document}
